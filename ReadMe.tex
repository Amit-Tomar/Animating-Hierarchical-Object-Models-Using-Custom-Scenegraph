\documentclass[11pt]{article}

\usepackage{xcolor}
\usepackage{listings}
\usepackage{graphicx}
\usepackage{subfigure}
\lstset{ %
language=C++,                % choose the language of the code
basicstyle=\footnotesize,       % the size of the fonts that are used for the code
backgroundcolor=\color{white!3!white},  % choose the background color. You must add \usepackage{color}
showspaces=false,               % show spaces adding particular underscores
showstringspaces=false,         % underline spaces within strings
showtabs=false,                 % show tabs within strings adding particular underscores
frame=single,           % adds a frame around the code
tabsize=2,          % sets default tabsize to 2 spaces
captionpos=b,           % sets the caption-position to bottom
breaklines=true,        % sets automatic line breaking
breakatwhitespace=false,    % sets if automatic breaks should only happen at whitespace
escapeinside={\%*}{*)},          % if you want to add a comment within your code
  basicstyle=\footnotesize\ttfamily,
  keywordstyle=\bfseries\color{green!40!black},
  commentstyle=\itshape\color{gray!80!black},
  identifierstyle=\color{black},
}


\title{\textbf{Computer Graphics : Assignment 6}}

\author{
		\vspace{ 2 mm}\\
		Supervised By : \textbf{Prof. Srikanth TK}\\
		\vspace {2mm}\\
		amit.tomar@iiitb.org \\
		(MT2013008) \\
		}
		
\date{2 - May - 2014}

\begin{document}
\lstset{language=C} 
\maketitle

\vspace{ 100 mm}

\section{Prerequisite Libraries}

\begin {enumerate}
\item OpenGl (Mesa libraries)
\item Qt5 libraries
\end {enumerate}

\section{Build Steps}

\begin {enumerate}
\item Change directory to folder MT2013008\_Assignment6/Src
\item Run the following command\\

\textbf{qmake MT2013008\_Assignment6.pro -o ../Build/Makefile} \\

This will generate the Makefile in the Build folder.

\item Change directory to folder MT2013008\_Assignment6/Build

\item Run the following command \\

\textbf{make} \\

This will build the complete project and generate and executable in the Build directory.

\item To start the application, go to build directory, execute following command 

\textbf{./MT2013008\_assignment6}



\end {enumerate}

\section{Usage}

\textbf{Note}: To enable keyboard controls, left-click on the left side of the screen.

\begin {enumerate}

\item Use 'q','e','w' keys to move arm 1.
\item Use 'a','s','d' keys to move arm 2.
\item Use 'z','x','c' keys to move arm 3.
\item Use 'i' key to make box move.
\item use 'p' key to move robot arm to belt 1.
\item use 'o' key to move robot arm to belt 2.
\item use 'k' key to pick box.
\item use 'l' key to drop box.

\end {enumerate}

\section{ Issues with implementation}

\begin {enumerate}
\item Implementing the program logic by adding more than one child to a node is giving some errors. More time to be spend on push-pop matrix concept.

\end {enumerate}             

\section{ Pending}

\begin {enumerate}
\item Multiple light sources.
\item Multiple view points.
\item Collision detection.
\end {enumerate}   

\end{document}

